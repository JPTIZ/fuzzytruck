%-------------------------------------------------------------------------------
\documentclass{article}

%-------------------------------------------------------------------------------
% Packages
\usepackage[portuguese]{babel}
\usepackage{environ}
\usepackage{hyperref}
\usepackage{minted}
\usepackage{xcolor}

%-------------------------------------------------------------------------------
% User-commands
\newcommand{\todo}[1]{{\color{red}{#1}}}

\NewEnviron{superframe}{%
    \begin{center}
        \fbox{\setlength{\fboxsep}{1em}\fbox{\parbox{5.5in}{%
            \BODY{}
        }}}
    \end{center}
}

\newmintedfile[textfile]{text}{autogobble, breaklines}

%-------------------------------------------------------------------------------
% Project configs
\title{Relatório de I.A.: Sistemas Fuzzy (Trabalho 4)}
\author{Cauê Baasch de Souza \\
        João Paulo Taylor Ienczak Zanette}
\date{\today}

%-------------------------------------------------------------------------------
\begin{document}
    \maketitle{}

    \todo{%
        TO-DO\@:
        \begin{itemize}
            \item Descrever tutorial breve de Fuzzy;
            \item Entradas;
            \item Conjuntos Fuzzy;
            \item Regras;
            \item Método de defuzzificação;
            \item Dificuldades encontradas e como foram superadas.
        \end{itemize}
    }

    \section{Introdução: Sistemas Fuzzy e o ``Fuzzy Truck''}

    Em um curso de Computação, é comum que se aprenda inicialmente a resolver
    problemas utilizando Lógica Proposicional, em que se julgam diferentes
    assertivas como ``verdadeiras'' ou ``falsas''. Tomando um exemplo simplista
    de previsão do tempo, pode-se partir das preposições:

    \begin{itemize}
        \item $h$: fez calor (\textit{``hot''});
        \item $c$: está nublado (\textit{``cloudy''});
        \item $r$: irá chover (\textit{``rain''}).
    \end{itemize}

    A partir delas, é possível estabelecer a relação ``se fez calor e está
    nublado, então irá chover'', ou seja:

    \begin{equation}
        h \land c \rightarrow r
    \end{equation}

    Porém, é de se concordar que fazer calor e estar nublado não são tão
    simples de se definir apenas com ``verdadeiro'' e ``falso'', mas sim como
    ``o quão quente'' e ``o quão nublado''. Esse é o ponto que Lógica Fuzzy
    ataca: a possibilidade de atribuir graus de verdade às assertivas dadas.
    Assim, é possível mapear que, dada a temperatura ao longo do dia (pode-se
    considerar, por exemplo, o ponto mais alto da temperatura), ela esteja mais
    próxima de fria quanto mais próxima de 0ºC, ou de quente quanto mais
    próxima de 40ºC, ou amena quanto mais próxima de 20ºC.



    \section{Resolvendo o problema do ``Fuzzy Truck''}

    \subsection{Entradas utilizadas}

    \subsection{Conjuntos Fuzzy}

    \subsection{Regras utilizadas}

    \subsection{Defuzzificação}

    \section{Conclusões e considerações}

    \bibliographystyle{unsrt}
    \bibliography{refs}
    \nocite{*}
\end{document}
