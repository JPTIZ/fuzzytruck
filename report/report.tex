%-------------------------------------------------------------------------------
\documentclass[twocolumn]{article}

%-------------------------------------------------------------------------------
% Packages
\usepackage[portuguese]{babel}
\usepackage{environ}
\usepackage[margin=2cm]{geometry}
\usepackage{graphicx}
\usepackage{hyperref}
\usepackage{minted}
\usepackage{xcolor}

%-------------------------------------------------------------------------------
% User-commands
\newcommand{\todo}[1]{{\color{red}{#1}}}

\NewEnviron{superframe}{%
    \begin{center}
        \fbox{\setlength{\fboxsep}{1em}\fbox{\parbox{5.5in}{%
            \BODY{}
        }}}
    \end{center}
}

\newmintedfile[textfile]{text}{autogobble, breaklines}

%-------------------------------------------------------------------------------
% Project configs
\title{Relatório de I.A.: Sistemas Fuzzy (Trabalho 4)}
\author{Cauê Baasch de Souza \\
        João Paulo Taylor Ienczak Zanette}
\date{\today}

%-------------------------------------------------------------------------------
\begin{document}
    \maketitle{}

    \todo{%
        TO-DO\@:
        \begin{itemize}
            \item Conjuntos Fuzzy;
            \item Regras;
            \item Método de defuzzificação;
            \item Dificuldades encontradas e como foram superadas.
        \end{itemize}
    }

    \section{O ``Fuzzy Truck''}

    No problema ``Fuzzy Truck'', é necessário definir parâmetros, saídas e
    regras de um sistema Fuzzy para fazer um caminhão, andando de ré,
    estacionar em uma doca considerando um espaço 2D sem obstáculos. Como
    restrições do problema, o caminhão possui velocidade constante (com exceção
    de que ele para quando está suficientemente próximo do espaço que delimita
    a vaga) e suas únicas ações possíveis são: girar o volante para a esquerda
    (indicado pelo valor -1), manter a direção atual do caminhão (valor 0) e
    girar o volante para a direita (valor 1).

    \section{Resolvendo o problema do ``Fuzzy Truck''}

    \subsection{Entradas utilizadas e Conjuntos Fuzzy}

    Foram utilizadas três entradas simples: as coordenadas $x$ e $y$ e o ângulo
    do caminhão (em graus). Os conjuntos fuzzy (visualizáveis na
    Figura~\ref{fuzzy-sets}) foram definidos pensando em:

    \begin{itemize}
        \item O quão longe o caminhão está horizontalmente, separando em:
            ``muito para a esquerda'', ``muito para a direita'' e
            ``centralizado'';
        \item O quão longe o caminhão está verticalmente, separando em: ``muito
            longe'' e ``próximo'';
        \item A direção do caminhão, separada em: norte, sul, leste e oeste.
    \end{itemize}

    \subsection{Regras utilizadas}

    \subsection{Defuzzificação}

    \section{Conclusões e considerações}

    \bibliographystyle{unsrt}
    \bibliography{refs}
    \nocite{*}
\end{document}
